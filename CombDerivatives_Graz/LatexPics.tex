% \documentclass[tikz,convert={outfile=\jobname.svg}]{standalone}
\documentclass[crop,equation,convert={outext=,command=\unexpanded{pdf2svg \infile\space ./LatexPics/LatexPic-\%d.svg all}},multi=alone]{standalone}
% \documentclass{article}
\usepackage{amsmath}
\usepackage{xcolor}

% Switch implementation https://tex.stackexchange.com/questions/64131/implementing-switch-cases
\usepackage{xifthen}
\newcommand{\ifequals}[3]{\ifthenelse{\equal{#1}{#2}}{#3}{}}
\newcommand{\case}[2]{#1 #2} % Dummy, so \renewcommand has something to overwrite...
\newenvironment{switch}[1]{\renewcommand{\case}{\ifequals{#1}}}{}

\usepackage[charter, cal=cmcal]{mathdesign}
\usepackage{mathtools,amsthm,booktabs,thmtools}
\usepackage{threeparttable}
\usepackage{numprint}
\usepackage{float} 
\usepackage{array}

\usepackage{xfrac}


\definecolor{darkorange}{rgb}{0.8, 0.45, 0}
\definecolor{darkgreen}{rgb}{0.0, 0.5, 0}
          
\usepackage{ytableau}

% \usepackage{longtable} % Split table over multiple pages; Flexible width column: tabularx
%\usepackage{ltablex} % Splitting of tables over multiple pages (longtable does not work with the X column)

%\allowdisplaybreaks % Allow breaking of align* over page breaks

% \usepackage[charter, cal=cmcal]{mathdesign} % Charter font with standard \mathcal
% \usepackage[charter]{mathdesign} % Charter font with standard \mathcal
% \usepackage{XCharter}

% \usepackage[garamond]{mathdesign}

%\renewcommand{\baselinestretch}{1.125} % Increase the distance between lines of text
% \renewcommand{\baselinestretch}{1.11} % Increase the distance between lines of text

% \usepackage[shortlabels]{enumitem} % Allows changing the enumerate labels, and changes the \ref labels for its items
%\usepackage[section]{placeins} % Keep all floats in their correct section; useful for notation tables
% \usepackage[commentmarkup=uwave]{changes} % Allows tracking of changes with \added, \deleted or \replaced
%\usepackage{setspace}\onehalfspacing % Increase line spacing

% \usepackage{quotchap}
%\newcommand{\quotewidth}{8cm}
% Header defintion. To use \thetitle, use package titling (before calling \title).
% \usepackage{fancyhdr}
% \setlength{\headheight}{26pt}
% \addtolength{\topmargin}{-10pt}

% \pagestyle{fancy}
% \fancyhead{}
% \fancyhead[LE,RO]{\thepage}
% \fancyhead[RE]{Symmetry reduction in convex optimization}
% \fancyhead[LO]{\nouppercase{\leftmark}}
% \fancyfoot{}
%\renewcommand{\headrulewidth}{0.4pt}
%\renewcommand{\headrulewidth}{0pt}

%\fancypagestyle{plain}{% Make the page number bold on the starting page of chapters
%	\fancyhf{}
%	\fancyfoot[C]{\bfseries\thepage}
%}

% \emergencystretch=1em

% \title{Symmetry reduction in convex optimization}
% \author{Daniel Brosch}

%TODO: Enable again, very slow!
% \usepackage{imakeidx}
% \makeindex[intoc]

\newcommand{\GraphPoly}[2]{M^{#1,#2}}

\usepackage{pgf,tikz,pgfplots}
\usetikzlibrary{shapes.geometric, graphs, fit, positioning}
\tikzset{
    gon1/.style={name=tmp,regular polygon,regular polygon sides=#1,minimum size=10pt,inner sep=(1/(2*(tan(pi/#1)))*0.15},
    % gon2/.style={name=tmp,regular polygon,regular polygon sides=#1,minimum size=8pt,inner sep=3pt},
    gon2/.style={name=tmp,regular polygon,regular polygon sides=#1,minimum size=10pt,inner sep=(1/(2*(tan(pi/#1)))*0.15},
    polygon side/.style args={#1--#2}{
        insert path={(tmp.corner #1)-- (tmp.corner #2)}
    }
}

\newcommand{\FlagGraph}[3][very thick,black,scale=1]{
\ifnum#2=2%
    %\tikz[baseline=(tmp1.south)]{
    \tikz[baseline=0pt,#1]{%(tmp1.center)]{
        \node[#1, circle,inner sep=2pt,fill] (tmp1) at (0,-4pt){};
        \node[#1,circle,inner sep=2pt,fill] (tmp2) at (0,12pt){};
        \ifx#3\empty%
        \else
            \draw[#1] (0,-4pt) -- (0,12pt);
        \fi
    }
\else%
    \tikz[baseline=-2pt]{%(tmp.south)]{    
    %\tikz[baseline=(tmp.south)]{
        \node[#1,gon1=#2]{};
        \foreach \X in {1,...,#2}{
            % \fill[#1] (tmp.corner \X) circle (1pt);
            \node [#1,circle,inner sep=2pt, fill] at (tmp.corner \X){};
            %\node [#1,circle,inner sep=0.7pt,fill] at (tmp.corner \X){};
        }
        \draw[#1,polygon side/.list={#3}];
    }
\fi
}

\newcommand{\FlagGraphNumb}[4][very thick, black, scale=1]{
\ifnum#2=2%
    \tikz[baseline=3pt]{%(tmp2.center)]{
        \ifx#3\empty%
        \else
            \draw[#1] (0,-4pt) -- (0,16pt);
        \fi
        \ifnum#4>0
          \node[#1, circle,inner sep=0.5pt, fill=white, draw=black] (tmp1) at (0,16pt){\footnotesize 1};
        \else
          \node[#1, circle,inner sep=2pt, fill] (tmp1) at (0,16pt){};
        \fi
        \ifnum#4>1
          \node[#1,circle,inner sep=0.5pt, fill=white, draw=black] (tmp2) at (0,-4pt){\footnotesize 2};
        \else 
          \node[#1,circle,inner sep=2pt, fill] (tmp2) at (0,-4pt){};
        \fi
    }
\else%
    \tikz[baseline=-2pt]{
        \node[#1,gon2=#2]{};
        % \node[name=tmp,regular polygon,regular polygon sides=#1,minimum size=8pt,inner sep=3pt]{};
        \draw[#1,polygon side/.list={#3}];
        \foreach \X in {1,...,#2}{
            %\fill (tmp.corner \X) circle (1pt);
            \pgfmathtruncatemacro\result{\X-1}
            \ifnum#4>\result
              \node [#1,circle=black,inner sep=0.5pt, fill=white, draw=black] at (tmp.corner \X){\footnotesize $\X$};
            \else 
              \node [#1,circle,inner sep=2pt, fill] at (tmp.corner \X){};
            \fi
        }

    }
\fi
}

\newcommand{\FlagGraphLabelThree}[5][very thick]{
  \tikz[baseline=-2pt]{
        \node[#1,gon2=3]{};
        % \node[name=tmp,regular polygon,regular polygon sides=#1,minimum size=8pt,inner sep=3pt]{};
        \draw[#1,polygon side/.list={#2}];
        
        \ifx#30
          \node [#1,circle,inner sep=2pt, fill] at (tmp.corner 1){};
        \else 
          \node [thick, circle,inner sep=1pt, fill=white, draw, minimum size = 3pt] at (tmp.corner 1){\footnotesize $#3$};
        \fi
        \ifx#40
          \node [#1,circle,inner sep=2pt, fill] at (tmp.corner 2){};
        \else 
          \node [thick, circle,inner sep=1pt, fill=white, draw, minimum size = 3pt] at (tmp.corner 2){\footnotesize $#4$};
        \fi
        \ifx#50
          \node [#1,circle,inner sep=2pt, fill] at (tmp.corner 3){};
        \else 
          \node [thick, circle,inner sep=1pt, fill=white, draw, minimum size = 3pt] at (tmp.corner 3){\footnotesize $#5$};
        \fi
    }
}

\usepackage{mathrsfs}
\usetikzlibrary{arrows}
\tikzset{
	triple/.style args={[#1] in [#2] in [#3]}{
		#1,preaction={preaction={draw,#3},draw,#2}
}}
\pgfplotsset{compat=newest}
\usetikzlibrary{intersections,calc,shapes,arrows,backgrounds,fit,matrix,positioning, decorations.pathreplacing}
\tikzset{>=stealth} % Default arrow tip
\tikzstyle{alg} = [rectangle, rounded corners, minimum height=1.5cm, text centered, draw=black, fill=gray!30, align=center, minimum width=0.5\textwidth] % For the blocks in the dependency graphs
\tikzstyle{algarrow} = [->, shorten >=1pt, thick] % For the arrows in the dependency graphs
% \tikzexternalize[figure name=_plot_sec\thesubsection_no,prefix=pics/] % \tikzexternalize should be after \makeindex
\usepgfplotslibrary{groupplots}
\pgfplotsset{
	discard if not/.style 2 args={
		x filter/.code={
			\edef\tempa{\thisrow{#1}}
			\edef\tempb{#2}
			\ifx\tempa\tempb
			\else
			\def\pgfmathresult{inf}
			\fi
		}
	},
%	every axis/.append style={font=\tiny}
}
\colorlet{nicegreen}{green!60!black}
\definecolor{xdxdff}{rgb}{0.49019607843137253,0.49019607843137253,1.}
\definecolor{ududff}{rgb}{0.30196078431372547,0.30196078431372547,1.}

% \newcommand{\Sym}{\text{\rm Sym}}

\def\pgfsysdriver{pgfsys-tex4ht.def}


\definecolor{ForestGreen}{RGB}{34,139,34}
\usetikzlibrary{arrows.meta}
\tikzset{
    side by side/.style 2 args={
    line width=2pt,
    #1,
    postaction={
        clip,postaction={draw,#2, line width=6pt}
        }
    }
}
\tikzstyle{treeNode}[black]=[draw=#1,fill=#1!20, circle, inner sep = 5pt]
\tikzstyle{treeNodeLabeled}[black]=[draw=#1, circle, inner sep = 1pt]
\tikzstyle{treeNodeMarked}[red]=[draw=#1,fill=#1!20, circle, inner sep = 5pt]%, edge=treeEdgeMarked]
\tikzstyle{treeNodeInner}=[inner sep = 0pt, circle, fill]
% \tikzstyle{treeNodeRoot}=[draw, inner sep = 3pt]
\tikzstyle{treeNodeRoot}=[inner sep = 2pt, circle, fill]
\tikzstyle{treeEdgeMarked}=[red, very thick]
\tikzstyle{treeEdgeDouble}=[side by side={red!50}{blue!50}]

% \usepackage[margins=0]{geometry}
\usepackage{ytableau}
\usepackage{forest}

\begin{document}
\color{black}

\forestset{
  default preamble={
      for tree={l = 0cm, s sep = 0.5cm,calign=fixed edge angles}
    }
}
\Large
% \boldmath
\begin{alone}
  \begin{forest}
    [, treeNodeRoot
        [, treeNode]
        [, treeNodeInner
            [, treeNodeInner
                [, treeNode]
                [, treeNode]
            ]
            [, treeNodeInner
                [, treeNode]
                [, treeNode]
            ]
        ]
    ]
  \end{forest}
\end{alone}
\begin{alone}
  \begin{forest}
    [, treeNodeRoot
        [, treeNodeMarked]
        [, treeNodeInner
            [, treeNodeInner
                [, treeNode]
                [, treeNodeMarked]
            ]
            [, treeNodeInner
                [, treeNodeMarked]
                [, treeNode]
            ]
        ]
    ]
  \end{forest}
\end{alone}
\begin{alone}
  \begin{forest}
    [, treeNodeRoot
    [, treeNodeMarked, edge=treeEdgeMarked]
    [, treeNodeInner, edge=treeEdgeMarked
    [, treeNodeInner, edge=treeEdgeMarked
    [, treeNode]
    [, treeNodeMarked, edge=treeEdgeMarked]
    ]
    [, treeNodeInner, edge=treeEdgeMarked
    [, treeNodeMarked, edge=treeEdgeMarked]
    [, treeNode]
    ]
    ]
    ]
  \end{forest}
\end{alone}
\begin{alone}
  \begin{forest}
    [, treeNodeRoot
    [, treeNodeMarked, edge=treeEdgeMarked]
    [, treeNodeInner, edge=treeEdgeMarked
    [, treeNodeInner, edge=treeEdgeMarked
    [, treeNodeMarked, edge=treeEdgeMarked]
    ]
    [, treeNodeInner, edge=treeEdgeMarked
    [, treeNodeMarked, edge=treeEdgeMarked]
    ]
    ]
    ]
  \end{forest}
\end{alone}
\begin{alone}
  \begin{forest}
    [, treeNodeRoot
        [, treeNode]
        [, treeNodeInner
            [, treeNode]
            [, treeNode]
        ]
    ]
  \end{forest}
\end{alone}
\begin{alone}
  $\mathrm{ex}\left(\FlagGraph[very thick,darkorange]{2}{1--2};\raisebox{-2pt}{\FlagGraph[very thick,darkgreen]{3}{1--2,2--3,1--3}}\right)\coloneqq \max_\mathcal{G} \left\lbrace \phi\left(\FlagGraph[very thick,darkorange,scale=1]{2}{1--2}\right) : \phi\left(\raisebox{-2pt}{\FlagGraph[very thick,darkgreen]{3}{1--2,2--3,1--3}}\right) = 0\right\rbrace.$
\end{alone}
\begin{alone}
  $\mathrm{ex}\left(\FlagGraph[very thick,darkorange]{2}{1--2};\raisebox{-2pt}{\FlagGraph[very thick,darkgreen]{3}{1--2,2--3,1--3}}\right) \geq \frac{1}{2},$
\end{alone}
\begin{alone}
  $\FlagGraph{2}{1--2}(\mathcal{G})  \cdot \raisebox{-2pt}{\FlagGraph{3}{1--2,2--3,1--3}}(\mathcal{G}) = \raisebox{-0pt}{\FlagGraph{5}{1--2,3--4,4--5,3--5}}(\mathcal{G}).$
\end{alone}
\begin{alone}
  $\FlagGraph{2}{1--2}\cdot\FlagGraph{3}{1--2,2--3,1--3}=  \FlagGraph{5}{1--2,3--4,4--5,3--5},$
\end{alone}
\begin{alone}
  $\FlagGraphNumb{3}{1--2,2--3,1--3}{2} \cdot \FlagGraphNumb{3}{1--2,1--3}{1} = \FlagGraphNumb{5}{1--2,1--3,2--3,1--4,1--5}{2}$
\end{alone}
\begin{alone}
  $\left(1 \varnothing - 2\FlagGraphNumb{2}{1--2}{1}\right)^2 = \varnothing -4\FlagGraphNumb{2}{1--2}{1}+4\FlagGraphNumb{3}{1--2,1--3}{1}\geq 0.$
\end{alone}
\begin{alone}
  $\left\llbracket \left(1 \varnothing - 2\FlagGraphNumb{2}{1--2}{1}\right)^2 \right\rrbracket = \left\llbracket \varnothing -4\FlagGraphNumb{2}{1--2}{1}+4\FlagGraphNumb{3}{1--2,1--3}{1}\right\rrbracket=\varnothing - 4\FlagGraph{2}{1--2} + 4\FlagGraph{3}{1--2,1--3}\geq 0.$
\end{alone}
\begin{alone}
  $\frac{1}{2} - \FlagGraph{2}{1--2} =  \left\llbracket \frac{1}{2} \left(1 \varnothing - 2\FlagGraphNumb{2}{1--2}{1}\right)^2 \right\rrbracket\\
    + \left\llbracket \left(\FlagGraphNumb{2}{1--2}{2} - \FlagGraphNumb{3}{1--2,2--3}{3} - \FlagGraphNumb{3}{1--2, 1--3}{3} + \FlagGraphNumb{3}{1--2,1--3,2--3}{3}\right)^2 \right\rrbracket\\
    -\FlagGraph{3}{1--2,1--3,2--3} \geq 0.$
\end{alone}
\begin{alone}
  $\llbracket f^2 \rrbracket = \langle cc^\top, \llbracket{\color{darkorange}\mathcal{F}}{\color{darkorange}\mathcal{F}}^\top\rrbracket\rangle$
\end{alone}
\begin{alone}
  $\langle M, \llbracket{\color{darkorange}\mathcal{F}}{\color{darkorange}\mathcal{F}}^\top\rrbracket\rangle$
\end{alone}
\begin{alone}
  $H - \FlagGraph{2}{1--2}^{|E(H)|} \geq 0.$
\end{alone}
\begin{alone}
  $\nabla_{\color{orange}v} f(x) = \lim_{{\color{red}h}\to 0}\frac{f(x+{\color{red}h}{\color{orange}v})-f(x)}{{\color{red}h}}$
\end{alone}
\begin{alone}
  $\partial_{\color{orange}v} H :=\lim_{{\color{red}i}\to\infty}\frac{(H \text{ density in
    }(G_{\color{red}i}-{\color{orange}v})) - (H \text{ density in }G_{\color{red}i})}{{\color{red}i}^{-1}}$
\end{alone}
\begin{alone}
  $\substack{\partial_1 \FlagGraph{2}{1--2} \enspace = \enspace 2\FlagGraph{2}{1--2} - 2\FlagGraphNumb{2}{1--2}{1}\\\vspace{1cm}\\ \partial_1 \FlagGraph{3}{1--2,1--3} \enspace = \enspace 3\FlagGraph{3}{1--2,1--3} - 2\FlagGraphLabelThree{1--2,1--3}{0}{1}{0} - \FlagGraphLabelThree{1--2,1--3}{1}{0}{0}}$
\end{alone}
\begin{alone}
  $\partial_{\color{orange}e} H :=\lim_{{\color{red}i}\to\infty}\frac{(H \text{ density in
    }(G_{\color{red}i}-{\color{orange}e})) - (H \text{ density in }G_{\color{red}i})}{{\color{red}i}^{-2}}$
\end{alone}
\begin{alone}
  $\substack{\partial_e \FlagGraph{2}{1--2} \enspace=\enspace - 2\FlagGraphNumb{2}{1--2}{2} \\\vspace{0.5cm}\\
      \partial_e \FlagGraph{3}{1--2,1--3} \enspace=\enspace - 2\FlagGraphLabelThree{1--2,1--3}{2}{1}{0} - 2\FlagGraphLabelThree{1--2,1--3}{1}{2}{0}
      \\\vspace{0.5cm}\\\partial_e \FlagGraph{3}{1--2,1--3,2--3} \enspace=\enspace - 6 \FlagGraphLabelThree{1--2,1--3,2--3}{1}{2}{0}}$
\end{alone}
\begin{alone}
  $\frac{1}{4}\left\llbracket\FlagGraphLabelTwo{1--2}{1}{3}\partial_{\text{swap}}(P_3-e^3)\right\rrbracket = \FlagGraph{6}{1--2,2--3,4--5,5--6} + \FlagGraph{6}{1--2,2--3,3--4,5--6}-2\FlagGraph{5}{1--2,2--3,2--5,4--5}\geq 0$

  % \bigg\llbracket \FlagGraphLabelTwo{1--2}{1}{3}\bigg(2\FlagGraphLabelFive{1--2, 3--4,4--5}{4}{3}{2}{0}{0}+2\FlagGraphLabelFive{1--2, 3--4,4--5}{4}{3}{1}{0}{0}+2\FlagGraphLabelFive{1--2, 3--4,4--5}{2}{1}{3}{0}{0}+2\FlagGraphLabelFive{1--2, 3--4,4--5}{2}{1}{4}{0}{0}+2\FlagGraphLabelSix{1--2,3--4,5--6}{4}{3}{1}{0}{0}{2}+2\FlagGraphLabelSix{1--2,3--4,5--6}{2}{1}{3}{4}{4}{4}\\\qquad
  % -2\FlagGraphLabelFour{1--2,2--3,3--4}{1}{2}{0}{0}-2\FlagGraphLabelFour{1--2,2--3,3--4}{2}{1}{0}{0}-2\FlagGraphLabelFour{1--2,2--3,3--4}{3}{4}{0}{0}-2\FlagGraphLabelFour{1--2,2--3,3--4}{4}{3}{0}{0}-2\FlagGraphLabelFour{1--2,2--3,3--4}{0}{1}{2}{0}-2\FlagGraphLabelFour{1--2,2--3,3--4}{0}{3}{4}{0}\bigg)\bigg\rrbracket$
\end{alone}
\begin{alone}
  But $\FlagGraph{5}{1--2,2--3,2--5,4--5} - \FlagGraph{6}{1--2,2--3,4--5,5--6} = \text{SOS}\geq 0$ and $\FlagGraph{5}{1--2,2--3,2--5,4--5} - \FlagGraph{6}{1--2,2--3,3--4,5--6}= \text{SOS}\geq 0$.
\end{alone}
\begin{alone}
  Thus $\FlagGraph{5}{1--2,2--3,2--5,4--5} = \FlagGraph{6}{1--2,2--3,4--5,5--6} = \FlagGraph{6}{1--2,2--3,3--4,5--6}$.
\end{alone}
\begin{alone}
  This proves $\FlagGraph{6}{1--2,2--3,3--4,5--6} = \FlagGraph{6}{1--2,2--3,4--5,5--6} \geq \FlagGraph{8}{1--2,3--4,5--6,7--8}$ by induction.
\end{alone}

\begin{alone}
  $\displaystyle\FlagGraphLabelThree{1--2, 1--3}{1}{0}{0}(\mathcal{G})\coloneqq \lim_{i\to\infty} \mathbb{P}\left[\sigma_i\left(\FlagGraphLabelThree{1--2, 1--3}{1}{0}{0}\right) \text{ is a subgraph of }G_i \mid \sigma_i(1) =1\right]$
\end{alone}


\begin{alone}
  $
    \begin{aligned}
      \partial_{\text{swap}}(P_3 - e^3) = & 2\FlagGraphLabelFive{1--2,3--4,4--5}{4}{3}{2}{0}{0} +2\FlagGraphLabelFive{1--2,3--4,4--5}{4}{3}{1}{0}{0}+2\FlagGraphLabelFive{1--2,3--4,4--5}{1}{2}{3}{0}{0}+2\FlagGraphLabelFive{1--2,3--4,4--5}{1}{2}{4}{0}{0} + 2\FlagGraphLabelSix{1--2,3--4, 5--6}{3}{4}{1}{0}{0}{2} + 2\FlagGraphLabelSix{1--2,3--4, 5--6}{1}{2}{3}{0}{0}{4} \\
                                          & -2\FlagGraphLabelFour{1--2,2--3,3--4}{1}{2}{0}{0}-2\FlagGraphLabelFour{1--2,2--3,3--4}{2}{1}{0}{0} -2\FlagGraphLabelFour{1--2,2--3,3--4}{3}{4}{0}{0}-2\FlagGraphLabelFour{1--2,2--3,3--4}{4}{3}{0}{0}-2\FlagGraphLabelFour{1--2,2--3,3--4}{0}{1}{2}{0}-2\FlagGraphLabelFour{1--2,2--3,3--4}{0}{3}{4}{0}
    \end{aligned}$

\end{alone}

\end{document}