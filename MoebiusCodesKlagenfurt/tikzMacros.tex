\usepackage{xcolor}

% Switch implementation https://tex.stackexchange.com/questions/64131/implementing-switch-cases
\usepackage{xifthen}
\newcommand{\ifequals}[3]{\ifthenelse{\equal{#1}{#2}}{#3}{}}
\newcommand{\case}[2]{#1 #2} % Dummy, so \renewcommand has something to overwrite...
\newenvironment{switch}[1]{\renewcommand{\case}{\ifequals{#1}}}{}

\usepackage[charter, cal=cmcal]{mathdesign}
\usepackage{mathtools,amsthm,booktabs,thmtools}
\usepackage{threeparttable}
\usepackage{numprint}
\usepackage{float} 
\usepackage{array}

\usepackage{xfrac}


\definecolor{darkorange}{rgb}{0.8, 0.45, 0}
\definecolor{darkgreen}{rgb}{0.0, 0.5, 0}
          
\usepackage{ytableau}

% \usepackage{longtable} % Split table over multiple pages; Flexible width column: tabularx
%\usepackage{ltablex} % Splitting of tables over multiple pages (longtable does not work with the X column)

%\allowdisplaybreaks % Allow breaking of align* over page breaks

% \usepackage[charter, cal=cmcal]{mathdesign} % Charter font with standard \mathcal
% \usepackage[charter]{mathdesign} % Charter font with standard \mathcal
% \usepackage{XCharter}

% \usepackage[garamond]{mathdesign}

%\renewcommand{\baselinestretch}{1.125} % Increase the distance between lines of text
% \renewcommand{\baselinestretch}{1.11} % Increase the distance between lines of text

% \usepackage[shortlabels]{enumitem} % Allows changing the enumerate labels, and changes the \ref labels for its items
%\usepackage[section]{placeins} % Keep all floats in their correct section; useful for notation tables
% \usepackage[commentmarkup=uwave]{changes} % Allows tracking of changes with \added, \deleted or \replaced
%\usepackage{setspace}\onehalfspacing % Increase line spacing

% \usepackage{quotchap}
%\newcommand{\quotewidth}{8cm}
% Header defintion. To use \thetitle, use package titling (before calling \title).
% \usepackage{fancyhdr}
% \setlength{\headheight}{26pt}
% \addtolength{\topmargin}{-10pt}

% \pagestyle{fancy}
% \fancyhead{}
% \fancyhead[LE,RO]{\thepage}
% \fancyhead[RE]{Symmetry reduction in convex optimization}
% \fancyhead[LO]{\nouppercase{\leftmark}}
% \fancyfoot{}
%\renewcommand{\headrulewidth}{0.4pt}
%\renewcommand{\headrulewidth}{0pt}

%\fancypagestyle{plain}{% Make the page number bold on the starting page of chapters
%	\fancyhf{}
%	\fancyfoot[C]{\bfseries\thepage}
%}

% \emergencystretch=1em

% \title{Symmetry reduction in convex optimization}
% \author{Daniel Brosch}

%TODO: Enable again, very slow!
% \usepackage{imakeidx}
% \makeindex[intoc]

\newcommand{\GraphPoly}[2]{M^{#1,#2}}

\usepackage{pgf,tikz,pgfplots}
\usetikzlibrary{shapes.geometric, graphs, fit, positioning}
\tikzset{
    gon1/.style={name=tmp,regular polygon,regular polygon sides=#1,minimum size=10pt,inner sep=(1/(2*(tan(pi/#1)))*0.15},
    % gon2/.style={name=tmp,regular polygon,regular polygon sides=#1,minimum size=8pt,inner sep=3pt},
    gon2/.style={name=tmp,regular polygon,regular polygon sides=#1,minimum size=10pt,inner sep=(1/(2*(tan(pi/#1)))*0.15},
    polygon side/.style args={#1--#2}{
        insert path={(tmp.corner #1)-- (tmp.corner #2)}
    }
}

\newcommand{\FlagGraph}[3][very thick,black,scale=1]{
\ifnum#2=2%
    %\tikz[baseline=(tmp1.south)]{
    \tikz[baseline=0pt,#1]{%(tmp1.center)]{
        \node[#1, circle,inner sep=2pt,fill] (tmp1) at (0,-4pt){};
        \node[#1,circle,inner sep=2pt,fill] (tmp2) at (0,12pt){};
        \ifx#3\empty%
        \else
            \draw[#1] (0,-4pt) -- (0,12pt);
        \fi
    }
\else%
    \tikz[baseline=-2pt]{%(tmp.south)]{    
    %\tikz[baseline=(tmp.south)]{
        \node[#1,gon1=#2]{};
        \foreach \X in {1,...,#2}{
            % \fill[#1] (tmp.corner \X) circle (1pt);
            \node [#1,circle,inner sep=2pt, fill] at (tmp.corner \X){};
            %\node [#1,circle,inner sep=0.7pt,fill] at (tmp.corner \X){};
        }
        \draw[#1,polygon side/.list={#3}];
    }
\fi
}

\newcommand{\FlagGraphNumb}[4][very thick, black, scale=1]{
\ifnum#2=2%
    \tikz[baseline=3pt]{%(tmp2.center)]{
        \ifx#3\empty%
        \else
            \draw[#1] (0,-4pt) -- (0,16pt);
        \fi
        \ifnum#4>0
          \node[#1, circle,inner sep=0.5pt, fill=white, draw=black] (tmp1) at (0,16pt){\footnotesize 1};
        \else
          \node[#1, circle,inner sep=2pt, fill] (tmp1) at (0,16pt){};
        \fi
        \ifnum#4>1
          \node[#1,circle,inner sep=0.5pt, fill=white, draw=black] (tmp2) at (0,-4pt){\footnotesize 2};
        \else 
          \node[#1,circle,inner sep=2pt, fill] (tmp2) at (0,-4pt){};
        \fi
    }
\else%
    \tikz[baseline=-2pt]{
        \node[#1,gon2=#2]{};
        % \node[name=tmp,regular polygon,regular polygon sides=#1,minimum size=8pt,inner sep=3pt]{};
        \draw[#1,polygon side/.list={#3}];
        \foreach \X in {1,...,#2}{
            %\fill (tmp.corner \X) circle (1pt);
            \pgfmathtruncatemacro\result{\X-1}
            \ifnum#4>\result
              \node [#1,circle=black,inner sep=0.5pt, fill=white, draw=black] at (tmp.corner \X){\footnotesize $\X$};
            \else 
              \node [#1,circle,inner sep=2pt, fill] at (tmp.corner \X){};
            \fi
        }

    }
\fi
}

\newcommand{\FlagGraphLabelThree}[5][very thick]{
  \tikz[baseline=-2pt]{
        \node[#1,gon2=3]{};
        % \node[name=tmp,regular polygon,regular polygon sides=#1,minimum size=8pt,inner sep=3pt]{};
        \draw[#1,polygon side/.list={#2}];
        
        \ifx#30
          \node [#1,circle,inner sep=2pt, fill] at (tmp.corner 1){};
        \else 
          \node [thick, circle,inner sep=1pt, fill=white, draw, minimum size = 3pt] at (tmp.corner 1){\footnotesize $#3$};
        \fi
        \ifx#40
          \node [#1,circle,inner sep=2pt, fill] at (tmp.corner 2){};
        \else 
          \node [thick, circle,inner sep=1pt, fill=white, draw, minimum size = 3pt] at (tmp.corner 2){\footnotesize $#4$};
        \fi
        \ifx#50
          \node [#1,circle,inner sep=2pt, fill] at (tmp.corner 3){};
        \else 
          \node [thick, circle,inner sep=1pt, fill=white, draw, minimum size = 3pt] at (tmp.corner 3){\footnotesize $#5$};
        \fi
    }
}

\usepackage{mathrsfs}
\usetikzlibrary{arrows}
\tikzset{
	triple/.style args={[#1] in [#2] in [#3]}{
		#1,preaction={preaction={draw,#3},draw,#2}
}}
\pgfplotsset{compat=newest}
\usetikzlibrary{intersections,calc,shapes,arrows,backgrounds,fit,matrix,positioning, decorations.pathreplacing}
\tikzset{>=stealth} % Default arrow tip
\tikzstyle{alg} = [rectangle, rounded corners, minimum height=1.5cm, text centered, draw=black, fill=gray!30, align=center, minimum width=0.5\textwidth] % For the blocks in the dependency graphs
\tikzstyle{algarrow} = [->, shorten >=1pt, thick] % For the arrows in the dependency graphs
% \tikzexternalize[figure name=_plot_sec\thesubsection_no,prefix=pics/] % \tikzexternalize should be after \makeindex
\usepgfplotslibrary{groupplots}
\pgfplotsset{
	discard if not/.style 2 args={
		x filter/.code={
			\edef\tempa{\thisrow{#1}}
			\edef\tempb{#2}
			\ifx\tempa\tempb
			\else
			\def\pgfmathresult{inf}
			\fi
		}
	},
%	every axis/.append style={font=\tiny}
}
\colorlet{nicegreen}{green!60!black}
\definecolor{xdxdff}{rgb}{0.49019607843137253,0.49019607843137253,1.}
\definecolor{ududff}{rgb}{0.30196078431372547,0.30196078431372547,1.}

% \newcommand{\Sym}{\text{\rm Sym}}

\def\pgfsysdriver{pgfsys-tex4ht.def}


\definecolor{ForestGreen}{RGB}{34,139,34}